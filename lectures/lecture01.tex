\section{Теория сложности}
 
Зачем доказывать трудность задач?
\begin{enumerate}
    \item Криптография.
    \item Нижние оценки требуют глубокого понимания (как доказать, что чего-то сделать нельзя?)
    \item Нижняя оценка может улучшать верхнюю оценку для другой задачи.
    \item Нижние оценки связаны с дерандомизацией.
\end{enumerate}
 
В теории сложности основная модель вычислений -- машина Тьюринга. Вспомним, что она представляет из себя.
 
\begin{example}
    Машина Тьюринга для проверки двоичной строки на палиндром. Пишется довольно очевидно, получаем количество шагов порядка $O(n^2)$. Но можно ли быстрее?
 
    Утверждается, что можно построить машину, работающую за $\frac{n^2}{C} + O(n)$, где $C$ -- сколь угодно большая константа. Идея: будем запоминать $C$ символов вместо одного, тогда количество шагов уменьшится в $C$ раз (хотя и коичество состояний увеличится).
 
    Можно доказать, что быстрее чем за квадратичное время нельзя.
\end{example}
 
\begin{theorem}
    Пусть $M$ -- одноленточная машина Тьюринга, распознающая язык бинарных палиндромов. Тогжа существует такая константа $C$ и $n_0$, что:
    \begin{gather*}
      \forall n > n_0 \; \exists x \in \{0, 1\}^n  : M(x) \text{ делает хотя бы $Cn^2$ шагов}
    \end{gather*}  
\end{theorem}
\begin{proof}
    Рассмотрим следующий тривиальный, но полезный принцип: 
    \begin{lemma}
      \textbf{Принцип несжимаемости}: пусть $f: \{0, 1\}^n \to \{0, 1\}^*$ -- инъективная функция. Тогда существует слово $x$ длины $n$, что $|f(x)| \geqslant n$.
    \end{lemma}
    \begin{proof}
      Проверяется очевидно исходя из того, что не может быть инъективного отображение в пространство меньшей размерности.
    \end{proof}

    Теперь будем рассматривать входы вида $x0^nx^{rev}$, где $x \in \{0, 1\}^n$ и $x^{rev}$ -- перевернутый $x$.
    Пусть $T(x)$ -- время работы машины на таком входе.
 
    Посмотрим на $n$ перегордок, которые идут после нулей.
 
    \textbf{Утверждение.} Существует перегородка, которую головка машины Тьюринга пересекала не более $\frac{T(x)}{n}$ раз. Доказывается очевидно.
 
    Введем функцию: 
    \begin{gather*}
      f: x \longmapsto (i, q_1, \dots, q_k)
    \end{gather*}
    где $i$ -- номер перегородки, которую пересекаем малое число раз, $q_1, \dots q_k$ -- последовательноcть состояний, в которых машина пересекла эту перегородку.
 
    \textbf{Утверждение.} $f$ инъективна. 
    \begin{proof}
      От противного: пусть $\exists x, y \in \{0, 1\}^n : f(x) = f(y)$. Посмотрим на строку $x0^ny^{rev}$. До пересечения $i$-ой перегородки машина не заметит разницу, при пересечении его мы попадем в такое состояние, пойдем левее -- такое же поведение, правее аналогично). В итоге мы дадим ответ Да, хотя это не палнидром, так как $x \neq y$.
    \end{proof} 
    Теперь по принципу несжимаемости существует такой $x \in \{0, 1\}^n$, что $|f(x)| >= n$. На $i$ нужно $C \log n$ битов. На стостояния нужно $k \cdot C$. Итого: $C\log n + k \cdot C >= n$. Добавляем ограничение на $k$ и получаем, что:
    \begin{gather*}
      T(x) >= \frac{n(n - C\log n)}{C} = \Omega(n^2)
    \end{gather*}
    Что и требовалось доказать.
\end{proof}
 
\begin{conj}
    Многоленточная машина Тьюринга. Есть $k$ одинаковых лент, вход будет написан только на первой ленте, головка есть на каждой. 
    Функция перехода теперь выглядит следующим образом:
    \begin{gather*}
      \delta: Q \times \Sigma^k \to Q \times \Sigma^k \times \{\leftarrow, \rightarrow, -\}^k
    \end{gather*}
\end{conj}
 
Очевидно, что на 2-ленточной машине можно решить задачу о палиндроме за $O(n)$. Скопируем слово на вторую ленту, на ней перейдем в конеw, и начнем идти двумя головками навстречу друг другу.
 
\textbf{Утверждение.} Для любой $k$-ленточной машины Тьюринга, которая на входе $x$ рабоатет $T(x)$, существует однолентчная, работающая на входе $x$ за $O(T(x)^2)$. 
\begin{proof}
  Надо увеличить алфавит, каждый символ будет хранить в себе сразу $k$ символов. \textbf{TODO}.
\end{proof}
 
\begin{theorem}
    Для любой $k$-ленточной машины Тьюринга, которая на входе $x$ работает $T(x)$ шагов, существует двуленточная, вычисляющая то же самое, и которая на входе $x$ работает $O(T(x)\log(T(x)))$ шагов.
\end{theorem}
\begin{proof}
    Ленту машины Тьюринга будем понимать как 2 стека вокруг головки. Итого у нас есть $2k$ стеков.
    Для начала научимся работать с одним стеком, возвращаясь в начало, а потом тот же трюк, что и в предыдущем док-ве.
    %На первой ленте будем хранить стек, а на второй делать операции со стеком.
 
    Более простая задача: ленивый счетчик. Есть операции $+1$ и $-1$. Хотим хранить значение так, чтобы при применении операции не приходилось менять в среднем константное колчество знаков. 
    Для этого используется ленивая двоичная система счисления: 
    \begin{gather*}
      c_0 + c_1 \cdot 2 + c_2 \cdot 4 + \dots \text{ , но } c_i \in \{0, 1, 2\}
    \end{gather*}

    Однозначности нет, но зато не надо будет менять много цифр. Считаем, что в минус не уходим. 
 
    Операция +1: находим первый разряд, который еще не переполнен -- $c_l < 2$. Делаем предыдущие разряды равными 1, а $c_l$ увеличиваем на 1.
 
    Операция -1: ищем первый непустой разряд -- $c_l > 0$. Делаем предыдущие равными $1$, а $c_l$ уменьшаем на 1.
 
    Между двумя операциями с $l$-тым разрядом происходит хотя бы $2^l$ шагов. Пусть теперь мы сделали $T$ операций со счетчиком и $f(l)$ -- операции с $l$-тым разрядом. Тогда \begin{gather*}
        \sum_{l=1}^\infty f(l) \leqslant \sum_{l=1}^\infty \frac{T}{2^l} = O(T)
    \end{gather*}
 
    Хранение стека: зоны, \textbf{картинка}.
 
    Для каждой зоны имеем: 0 -- пустая, 1 -- заполнена наполовину, 2 -- заполнена полностью. Получаем ленивую систему счисления. Тогда при добавлении / удалении элемента зоны не будут сильно сдвигаться.
 
    Добавление элемента в стек. Пусть $l$-тая зона -- первая, не полностью заполненная. Пока ищем, все элементы записываем на вторую ленту. Как только нашли, пойдем назад и будем раскидывать элементы со второй ленты в нужные зоны.
 
    Удаление элемента из стека. Пусть $l$-тая зона -- первая непустая. Удалим ее первый элемент и все предыдущие зоны наполним. Реазизуем опять же с помощью второй ленты.
 
    Сколько операций мы тратим? Если затронули $l$-тую зону, то выполнили где-то $2^l$ операций. Пусть было $T$ операция со стеком и $f(l)$ -- число операция с $l$-той зоной: \begin{gather*}
        \sum_{l = 1}^k f(l) \cdot O(2^l) \leqslant \sum_{l = 1}^k \frac{T}{2^l} \cdot O(2^l) = O(k \cdot T) 
    \end{gather*}
    Легко видеть, что $k \leqslant O(\log T)$. Что и требовалось доказать.
 
\end{proof}